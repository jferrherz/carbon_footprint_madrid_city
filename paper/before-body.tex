$if(has-frontmatter)$
\frontmatter
$endif$

$if(title)$
\makeatletter
\twocolumn[
\maketitle

\begin{@twocolumnfalse}

\centering\begin{minipage}{\dimexpr\paperwidth-6.5cm}
\begin{abstract}
\begin{normalsize}
% Write your abstract here ----------------------------------------------------
\vspace{1em}
Urban economies bear a significant responsibility in climate change, but estimating scope-3 emissions of urban areas is challenging due to the lack of city-level input-output tables and air emissions accounts. These limitations have led to various bottom-up and top-down strategies to address the information gap, often relying on the adoption of stringent domestic input-output modeling assumptions that limit considerably the analytical scope and granularity of the results. This paper presents the most rigorous quantification of the 3-scope carbon footprint of the city of Madrid to date, a full 2010-2021 series with a 28-industry and 47-country breakdown. The methodology consists in, firstly, integrating municipal-level economic data with regional supply and use tables to construct a series of city-specific input-output tables and to incorporate them into the FIGARO global multiregional input-output framework; secondly, constructing the first air emissions account for the city of Madrid, including direct emissions by households; thirdly,deriving a final consumption expenditure vector from household surveys. Our analysis reveals that the city's carbon footprint in 2021 was 17,447 ktCO2e for GDP and 13,920 ktCO2e for households. Only 15\% originated within the city limits, 34\% from the rest of Spain, and 51\% from the rest of the world. The analytical possibilities offered by these data are considerably broad. We find significant emissions inequality among residents, with the top 20\% of the equivalized spending distribution emitting 4.8 times more than the bottom 20\%. %
Structural decomposition analysis shows that while efficiency improvements and technological advancements contributed to emissions reduction, these gains were largely offset by increases in consumption demand. 
Notably, our simulations indicate that reshaping consumption patterns of higher-income households could potentially reduce emissions by up to 26\%, which is similar to the potential emissions savings from a substantial shift modality in private transport. The study underscores the need for consumption-oriented mitigation strategies and highlights the importance of addressing emissions inequality in urban climate action plans.
\end{normalsize}
\end{abstract}
$if(keywords)$
\begin{flushleft}
\setlength\parindent{30pt}{
\hspace{\parindent}\small\textbf{Keywords}: {$for(keywords)$$keywords$$sep$, $endfor$}.}
\end{flushleft}
$endif$
\end{minipage}

$if(JEL)$
\centering\begin{minipage}{\dimexpr\paperwidth-8cm}
\begin{flushleft}
{\setlength\parindent{30pt}
\hspace{\parindent}\small\textbf{JEL classification}: {$for(JEL)$$JEL$$sep$, $endfor$}.}
\end{flushleft}
\end{minipage}
$endif$
\vspace{3em}
\end{@twocolumnfalse}]
\makeatother
$endif$